\subsection{10 августа. Пер. Иттиш}
\textit{Метеоусловия: утром, днём, вечером ясно, тепло. В середине дня жарко.}

\begin{figure}[h!]
	\centering
		\includegraphics[angle=0, width=0.7\linewidth]{pics/maps/10}
	\label{fig:mini_10}
\end{figure}

Выходим в \alert{07:30} с м.н. 6. К \alert{10:30} преодолеваем подъем по морене и выходим на горизонтальную осыпь на высоте $\sim$3700м. Оставшуюся часть дня двигаемся по этой осыпи.

\begin{figure}[h!]
	\centering
	\includegraphics[width=0.7\linewidth]{pics/10/IMG_3429}
	\caption{Осыпь на подходе к пер. Иттиш}
	\label{fig:IMG_3429}
\end{figure}

Горизонтальная осыпь состоит из средних и крупных камней. В некоторых местах двигаемся по травянистым участкам, в некоторых -- по пересохшей части дна озер, в некоторых приходится двигаться лазанием по крупным камням. Для ускорения рекомендуем ориентироваться на установленные туры, избегать лазания, пользоваться движением по дну озер, где это возможно. Однозначного рецепта преодоления данной осыпи нет.

\begin{figure}[h!]
	\centering
	\includegraphics[width=0.7\linewidth]{pics/10/IMG_3421}
	\caption{Движение по пересохшему дну озера}
	\label{fig:IMG_3421}
\end{figure}

В самое жаркое время суток, с 13:00 до 15:00, устраиваем обед вблизи истока ручья, впадающего в одно из озер. После этого прошли часть пути по дну озера и далее двигались по гребням морен. Морены состоят из камней среднего размера.

\begin{figure}[h!]
	\centering
	\includegraphics[width=0.7\linewidth]{pics/10/IMG_3489}
	\caption{Группа на седловине пер. Иттиш. Вид на сырты и хр. Акшийрак.}
	\label{fig:IMG_3489}
\end{figure}

На седловину перевала вышли в \alert{17:15}, преодолев кулуар, разделяющий моренный гребень с седловиной перевала. Спуска с перевала нет, сразу идет выход на высокогорные луга -- сырты. Пройдя по горизонтали около 1км, встаем на ночевку на озере в 18:00, соединившись с группой Екатерины Тюриной. Координаты м.н.: \alert{N\degree, E\degree}.

\begin{figure}[h!]
	\centering
		\includegraphics[width=0.7\linewidth]{pics/10/camp_10}
	\caption{Место ночёвки 10-11.08}
	\label{fig:camp_10}
\end{figure}

\clearpage