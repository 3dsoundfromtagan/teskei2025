\subsection{Пер. Иттиш}

Пер. Иттиш ориентирован с севера на юг и является перевалом через основной хр. Тескей-Ала-Тоо. Это один из немногих перевалов через данный основной хребет, которые имеют (низкую) трудность 1А. Вообще говоря, пер. Иттиш - это выход на горизонтальное плато, на котором расположены сырты (заболоченные луга). Лишь северная сторона перевала имеет наклон. Ранее \alert{ссылка} перевал проходился на спуск с юга на север; в данном походе было решено преодолеть его на подъем с севера на юг. При подготовке использовались отчеты \alert{ссылки}.

Подъем на перевал можно разбить на две части: подъем по тропе по хвойному лесу и осыпи и горизонтальное движение по средней и крупной осыпи. Второй этап недлинный по расстоянию -- около 4-5 км -- но может занять много времени из-за трудности рельефа. В некоторых местах можно сойти с крупных камней и двигаться по пересохшему дну озер, находящихся вблизи осыпи. При подходе к седловине видно обледенелый пик Иттиш и ледник, примыкающий к нему. После взятия перевала спуска нет -- сразу идет выход на высокогорные сырты и озера. Также в хорошую погоду с седловины открывается вид на соседний хр. Акшийрак. В целом, подъем является постепенным и не имеет крутых взлетов.

Суммарный перепад высот от д.р. Джууку, из которой начинался подъем, до седловины перевала составляет около 1000м, поэтому было решено устроить ночевку в середине подъема на небольшом озере, находящимся выше границы леса. От м.н. ночевки на озере до седловины перевала был пройден остаток подъема по средней осыпи и затем преодолен упомянутый затяжной участок горизонтальной осыпи.