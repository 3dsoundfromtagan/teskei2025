\begin{titlepage}
\setlength\parindent{0pt}
	\begin{center}
		\large{РОО <<Федерация спортивного туризма Московской области>>\\
		ОО г. Долгопрудного <<Федерация спортивного туризма>>\\}
	\end{center}

	


	
	\begin{center}
		\includegraphics[width=0.4\linewidth]{pics/Flag_GS-2}
		
		\Large{\bfseries{ОТЧЁТ}} \\
		\normalsize о прохождении горного спортивного туристического маршрута \textbf{первой} категории сложности по Центральному Тянь-Шаню (хребе Тескей-Ала-Тоо), совершённом с 03 по 13 августа 2025 г. группой туристов Горной секции МФТИ ФСТ Московской области, г. Долгопрудный
	\end{center}
	\vspace{1.5 cm}
	
	\textbf{Маршрутная книжка:} 55/2025, выдана МКК ФСТ МО г. Долгопрудный \\ 
	\textbf{Руководитель группы:} Остапив Алексей Юрьевич\\
	\textbf{E-mail:} \href{mailto: ostapiv\_ayu@phystech.edu}{ostapiv\_ayu@phystech.edu}\\
	\textbf{Номер телефона:} $+7(989)629-46-58$
	
	\vspace{0.2cm}
	
	\textit{Маршрутно-квалификационная комиссия Федерации спортивного туризма Московской области рассмотрела отчёт и считает, что маршрут может быть зачтен всем участникам и руководителю \textbf{первой категории сложности}.}

	\vspace{0.2cm}
	
	Отчёт использовать в библиотеке ФСТ Московской области и ФСТ г. Долгопрудный.
	
	\vspace{0.8cm}
	\textbf{Судья по виду:} 
	
%	\vspace{0.8cm}
%	\textbf{Судья по ходу:}
	
	\vspace{0.8cm}
	\textbf{Председатель МКК:}
	
	\vspace{0.8cm}
	\textbf{Штамп МКК:}
	
	\vfill
	\begin{center}
		Долгопрудный,   \the\year{}
	\end{center}
\end{titlepage}