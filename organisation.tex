\section{Организация и проведение похода}
\subsection{Цели и задачи маршрута. Выбор нитки маршрута}
При разработке и планировании маршрута я, как руководитель, руководствовался следующими соображениями:
\begin{enumerate} 
	\item \textbf{Высотность, транспортная доступность горного района и стоимость трансфера.}
	Подавляющая часть группы (шесть человек из восьми) имела официальный или неофициальный опыт горных походов до 2 к.с по Кавказу \cite{Snegovskaya2024} или Алтаю, и вся группа имела высотный опыт ночёвки не менее 2900 м \cite{ostapiv2025}. Это позволило выбрать в качестве района новичкового похода не Гвандру или Архыз, как это обычно бывает, а новый для участников и руководителя горный район Центральной Азии~--- Тескей-Ала-Тоо. Высоты в этом районе, в среднем, на 500 м больше, чем на Кавказе, а его транспортную доступность можно назвать приемлемой для новичковых походов: 4 часа самолётом до Бишкека и еще 7~--- до места старта в горах. Стоимость трансфера выше, чем на Кавказе, но ниже, чем на Алтае. 
		
	\item \textbf{Концентрированность препятствий.}
	Дополнительным фактором в сторону выбора Тескей-Ала-Тоо послужило также и то, что в отличие, например, от Алтая, хребтовка района (основной хребет и серия параллельных отрогов), как и на Кавказе, позволяет осуществить подход под многие перевалы в течение одного дня, миновав долгие и занудные забеги по долинам, и, как следствие, поддерживать интерес группы на приемлемом уровне.
	
	\item \textbf{Разнообразие рельефа.} 
	С методической точки зрения, а также, опять-таки, для поддержания интереса группы, хотелось продемонстрировать участникам как можно больше разнообразных типов рельефа. Район позволяет в походах 1 к.с. продемонстрировать все виды рельефа, за исключением, пожалуй, снежного. <<Изюминкой>> района является возможность пересечения главного хребта через перевалы 1А (разительно отличающихся от классических <<единичек А>>) с осмотром высокогорных болот~--- сыртов.
	
	\item \textbf{Эффект кульминации.}
	У меня как у руководителя было глубокое убеждение, что первый поход должен обладать понятным, с позволения сказать, сюжетом и иметь свою кульминацию. В нашем случае таким сюжетом было движение через отроги главного хребта с преодолением разнообразных перевалов (последовательно: травянисто-осыпной, ледовый, травянистый), а затем~--- кольцо через главный хребет с <<неклассическими>> перевалами на южную сторону, восхождением на обзорную точку~--- вершину Марс. Конец маршрута~--- забег по среднегорью с финишем на локальной достопримечательности <<Красные скалы>>~--- также должен был добавить в поход разнообразия и красоты. Наконец, по моему глубокому убеждению, финальным аккордом походов по Кыргызстану должен быть хотя бы  однодневный (а лучше двухдневный) отдых на Иссык-Куле.
	
	\item\textbf{Помощь в планировании}
	Фактор, который формально не был в списке определяющих критериев, но по факту являлся таковым~--- это искренняя заинтересованность и помощь в планировании и организации маршрута от человека, через которого организовывалась логистика похода~--- Юрия Траченко. За эту помощь выражаю ему огромную благодарность.
	
\end{enumerate} 
\subsection{Логистика}
Подъезд осуществлён на поезде 033М Москва--Владикавказ до станции Минеральные Воды (прибытие в 03:40). Стоимость проезда на август 2024 г. составляла 7800~\faRub, купе (обратно – 4700~\faRub, плацкарт). От Минеральных Вод до аула Верхний Учкулан (время в пути 4 часа) добирались на трансфере, заказанном через Саракуева Бориса (89289503868, 89298843175,  \href{mailto: bezonec@list.ru}{bezonec@list.ru}). Стоимость трансфера трансфера туда составила 18000~\faRub, обратно (от поляны Азау) — 15000~\faRub. Стоимости доставки забросок в т/б Глобус и а/л Узункол составили 4000~\faRub~и 6000~\faRub.
Коллективный пропуск в пограничную зону КЧР был оформлен за 4 месяца до начала похода через электронную почту пограничного управления ФСБ по КЧР~--- \href{mailto: pu.kcherkes@fsb.ru}{pu.kcherkes@fsb.ru} и отправлен письмом по указанному адресу. Пропуск в КБР не требуется, так как пер. Хотютау в 2023 году был исключён из пограничной зоны \cite{order_kbr}.
\subsection{Аварийные выходы из маршрута и его запасные варианты}
\textbf{Аварийными выходами} с маршрута являлись:
\begin{itemize}
	\item На первом этапе: спуск к т/б <<Глобус>>;
	\item На втором этапе: спуск к а/л <<Узункол>>;
	\item На третьем этапе: спуск к погранзаставе <<Актюбе>> (Хурзук).
\end{itemize}
\textbf{Запасными вариантами} маршрута являлись:
\begin{itemize}
	\item Замена пер. Уллу-Кёль Восточный (1А$^\star$, 3050) на пер. \textbf{Уллу-Кёль Нижний (н/к, 2933)};
	\item Отказ от пер. Перемётный (1А, 3255), спуск по д.р. Чунгур-Джар;
	\item Отказ от пер. Хотютау (1А$^\star$), спуск по д.р. Кубань к погранзаставе <<Хурзук>>
\end{itemize}
\subsection{Изменение маршрута и их причины}
Маршрут пройден без изменений.
\subsection{Обеспечение безопасности на маршруте}
Группа была зарегистрирована в отделении МЧС Кыргызстана по Иссык-Кульской области (телефон +996555004214, связь по WhatsApp). За 10 дней до похода были переданы сведения о составе группы, сроках и маршруте, в ответ был получен регистрационный номер и просьба проинформировать дежурного о начале и завершении похода.

\alert{Дима, это с тебя Для регулярного обмена сообщениями, отслеживания положения группы на карте, а также возможности экстренной связи, в группе имелся спутниковый треккер IRIDIUM Rockstar 360. Стоимость аренды треккера в <<Альпиндустрии>> на 15--21 день составила 7100~\faRub, залог~--- 50000~\faRub. Нам повезло попасть на демострационный период тарифа треккера, в связи с чем все сообщения были для нас безлимитны и бесплатны. Предварительное тестирование треккера в Москве показало, что спутниковые сигналы в столице эффективно глушатся: сообщения приходили не чаще раза в сутки. В походе с приёмом и отправкой сообщений и координат на сервер проблем не возникало, среднее время отправки составляло 30 минут.}

Каждый участник самостоятельно оформлял на себя индивилуальный страховой полис. Выбрали страховую фирму <<Согласие>>, ассист Balt Assistance, опция <<Треккинг свыше 1500 м>>, размер страховой защиты 35000~USD,  вид отдыха <<Спорт Экстрим>>. Стоимость полиса составила 8234~\faRuble~с человека.

\subsection{Перечень наиболее интересных природных и исторических объектов, занятий на маршруте}
\begin{enumerate}[noitemsep,topsep=0pt,parsep=0pt,partopsep=0pt]
	\item Широкие красивые долины северной стороны хребта: Чон-Кызыл-Суу, Киче-Кызыл-Суу, Ашукашкасу, Джукучак; 
	\item Сырты на южной стороне хребта; 
	\item Цепочка озёр Кашкасу с чистой водой; 
	\item Вершина Марс, с которой открываются виды на Акширак, Кумтор, сырты и, в хорошую погода, на семитысячники; 
	\item Перевал Кашкасу, через который в былые годы перегоняли лошадей. Следы этого остаются в ущелье до сих пор, жутковато и атмосферно; 
	\item <<Красные скалы>> на слиянии р. Джууку и Джукучак;
	\item Иссык-Куль.
\end{enumerate}

\paragraph{Темы практических занятий:}

\begin{itemize}
	\item Техника передвижения по травянистым и осыпным склонам;
	\item Техника несложных бродов поодиночке;
	\item Техника передвижения по льду.
\end{itemize}

\newpage
\subsection{Развёрнутый график движения}
\begin{table}[h!]
	\centering
	\resizebox{0.95\textwidth}{!}{%
		\begin{tabular}{|>{\centering\arraybackslash}m{0.045\linewidth}
				|>{\centering\arraybackslash}m{0.02\linewidth}
				|>{\centering\arraybackslash}m{0.43\linewidth}
				|>{\centering\arraybackslash}m{0.09\linewidth}
				|>{\centering\arraybackslash}m{0.1\linewidth}
				|>{\centering\arraybackslash}m{0.05\linewidth}
				|>{\centering\arraybackslash}m{0.09\linewidth}
				|>{\centering\arraybackslash}m{0.13\linewidth}|}
			\hline						
			Дата	&	\begin{turn}{90}День\end{turn}	&	Участок маршрута	&	Км с $k=1.2$	&	Набор /сброс, м	&	ЧХВ	&	Высота ночёвки, м	&	Способы передвижения	\\
			\hline
			
			18.08	&	1	&	г.~Минеральные воды~--- аул Верхний Учкулан~--- д.р Учкулан~--- д.р. Кичкинакол Уллукёльский	&	5.3	&	$+650$\newline$-0$	& 2:46	&	2200	&	Машина,\newline Пешком	\\
			\hline
			19.08	&	2	&	д.р. Кичкинакол Уллукёльский~--- оз. Гитче-Кёль~--- оз. Уллу-Кёль 	&	5.6	& $+650$\newline$-0$		& 3:25		& 2850		&	Пешком	\\
			\hline
			20.08	&	3	&	м.н.~--- \textbf{пер. Уллу-Кёль Восточный (1А$^\star$, 3050)}~--- кош в д.р. Трёхозёрная~--- д.р. Махар	&	7.2	& $+200$\newline$-1190$		& 7:39	& 1860		&	Пешком	\\
			\hline
			21.08	&	4	&	м.н.~--- т/б <<Глобус>>~--- д.р. Гондарай~--- д.р. Джалпаккол	&	11.3	&$+390$\newline$-225$		& 3:54		& 2120		&	Пешком	\\
			\hline
			22.08	&	5	&	м.н.~--- д.р. Кичкинекол Джалпаккольский~--- м.н. под моренным валом пер. Джалпаккол Северный	&	5.8	& $+620$\newline$-0$		& 3:56	& 2740		&	Пешком	\\
			\hline
			23.08	&	6	&	м.н.~--- \textbf{пер. Джалпаккол Северный (1А$^\star$, 3411)}~--- зелёные ночёвки на спуске в д.р. Мырды	&	5.0 	& $+660$\newline$-395$		& 6:16		& 3015		&	Пешком	\\
			\hline
			24.08	&	7	&	м.н.~--- д.р. Мырды~--- а/л <<Узункол>>	&	7.5	& $+0$\newline$-960$		& 3:53		& 2060		&	Пешком	\\
			\hline
			25.08	&	8	&	м.н.~--- д.р. Кичкинекол~--- д.р. Таллычат~--- Поляна Крокусов	&	7.1	& $+780$\newline$-0$		& 3:23		& 2840		&	Пешком	\\
			\hline
			26.08	&	9	&	м.н.~--- \textbf{пер. Кичкинекол Малый (1А, 3206)}~--- д.р. Чунгур-Джар	&	4.6	& $+360$\newline$-520$		& 2:42		& 2680		&	Пешком	\\
			\hline
			27.08	&	10	&	м.н.~--- \textbf{пер. Перемётный (1А, 3255)}~--- д.р. Танышхан	&	7.1	& $+575$\newline$-935$		& 6:50		& 2320		&	Пешком	\\
			\hline
			28.08	&	11	&	м.н.~--- д.р. Чиринкол~--- д.р. Кубань &	12.7	& $+90$\newline$-500$		& 3:23		& 1890		&	Пешком	\\
			\hline
			29.08	&	12	&	м.н.~--- погранзастава <<Хурзук>>~(рад.)~--- д.р. Уллу-Кам	&	20.9	& $+1210$\newline$-370$		& 7:15		& 2725		&	Пешком	\\
			\hline
			30.08	&	13	&	м.н.~--- \textbf{пер. Хотютау (1А$^\star$, 3546)}~--- лед. Большой Азау~--- оз. Эльбрусское~--- ст. <<Старый Кругозор>>~--- поляна Азау & 10.9	& $+800$\newline$-615$		& 4:25		& 2915		&	Пешком, Канатная дорога	\\
			\hline
			\multicolumn{3}{|c|}{\textbf{\textit{\Large{Итого:}}}} & \large{\textbf{111.0}} & \large{$\mathbf{+6985}$\newline$\mathbf{-5210}$	}	& \multicolumn{3}{c|}{\large{\textbf{58:08}\newline\textbf{2д 10ч 08мин}}} \\
			\hline
		\end{tabular}
}	
	
\end{table}



\clearpage